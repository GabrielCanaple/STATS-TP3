% Options for packages loaded elsewhere
\PassOptionsToPackage{unicode}{hyperref}
\PassOptionsToPackage{hyphens}{url}
%
\documentclass[
]{article}
\usepackage{amsmath,amssymb}
\usepackage{lmodern}
\usepackage{iftex}
\ifPDFTeX
  \usepackage[T1]{fontenc}
  \usepackage[utf8]{inputenc}
  \usepackage{textcomp} % provide euro and other symbols
\else % if luatex or xetex
  \usepackage{unicode-math}
  \defaultfontfeatures{Scale=MatchLowercase}
  \defaultfontfeatures[\rmfamily]{Ligatures=TeX,Scale=1}
\fi
% Use upquote if available, for straight quotes in verbatim environments
\IfFileExists{upquote.sty}{\usepackage{upquote}}{}
\IfFileExists{microtype.sty}{% use microtype if available
  \usepackage[]{microtype}
  \UseMicrotypeSet[protrusion]{basicmath} % disable protrusion for tt fonts
}{}
\makeatletter
\@ifundefined{KOMAClassName}{% if non-KOMA class
  \IfFileExists{parskip.sty}{%
    \usepackage{parskip}
  }{% else
    \setlength{\parindent}{0pt}
    \setlength{\parskip}{6pt plus 2pt minus 1pt}}
}{% if KOMA class
  \KOMAoptions{parskip=half}}
\makeatother
\usepackage{xcolor}
\usepackage[margin=1in]{geometry}
\usepackage{color}
\usepackage{fancyvrb}
\newcommand{\VerbBar}{|}
\newcommand{\VERB}{\Verb[commandchars=\\\{\}]}
\DefineVerbatimEnvironment{Highlighting}{Verbatim}{commandchars=\\\{\}}
% Add ',fontsize=\small' for more characters per line
\usepackage{framed}
\definecolor{shadecolor}{RGB}{248,248,248}
\newenvironment{Shaded}{\begin{snugshade}}{\end{snugshade}}
\newcommand{\AlertTok}[1]{\textcolor[rgb]{0.94,0.16,0.16}{#1}}
\newcommand{\AnnotationTok}[1]{\textcolor[rgb]{0.56,0.35,0.01}{\textbf{\textit{#1}}}}
\newcommand{\AttributeTok}[1]{\textcolor[rgb]{0.77,0.63,0.00}{#1}}
\newcommand{\BaseNTok}[1]{\textcolor[rgb]{0.00,0.00,0.81}{#1}}
\newcommand{\BuiltInTok}[1]{#1}
\newcommand{\CharTok}[1]{\textcolor[rgb]{0.31,0.60,0.02}{#1}}
\newcommand{\CommentTok}[1]{\textcolor[rgb]{0.56,0.35,0.01}{\textit{#1}}}
\newcommand{\CommentVarTok}[1]{\textcolor[rgb]{0.56,0.35,0.01}{\textbf{\textit{#1}}}}
\newcommand{\ConstantTok}[1]{\textcolor[rgb]{0.00,0.00,0.00}{#1}}
\newcommand{\ControlFlowTok}[1]{\textcolor[rgb]{0.13,0.29,0.53}{\textbf{#1}}}
\newcommand{\DataTypeTok}[1]{\textcolor[rgb]{0.13,0.29,0.53}{#1}}
\newcommand{\DecValTok}[1]{\textcolor[rgb]{0.00,0.00,0.81}{#1}}
\newcommand{\DocumentationTok}[1]{\textcolor[rgb]{0.56,0.35,0.01}{\textbf{\textit{#1}}}}
\newcommand{\ErrorTok}[1]{\textcolor[rgb]{0.64,0.00,0.00}{\textbf{#1}}}
\newcommand{\ExtensionTok}[1]{#1}
\newcommand{\FloatTok}[1]{\textcolor[rgb]{0.00,0.00,0.81}{#1}}
\newcommand{\FunctionTok}[1]{\textcolor[rgb]{0.00,0.00,0.00}{#1}}
\newcommand{\ImportTok}[1]{#1}
\newcommand{\InformationTok}[1]{\textcolor[rgb]{0.56,0.35,0.01}{\textbf{\textit{#1}}}}
\newcommand{\KeywordTok}[1]{\textcolor[rgb]{0.13,0.29,0.53}{\textbf{#1}}}
\newcommand{\NormalTok}[1]{#1}
\newcommand{\OperatorTok}[1]{\textcolor[rgb]{0.81,0.36,0.00}{\textbf{#1}}}
\newcommand{\OtherTok}[1]{\textcolor[rgb]{0.56,0.35,0.01}{#1}}
\newcommand{\PreprocessorTok}[1]{\textcolor[rgb]{0.56,0.35,0.01}{\textit{#1}}}
\newcommand{\RegionMarkerTok}[1]{#1}
\newcommand{\SpecialCharTok}[1]{\textcolor[rgb]{0.00,0.00,0.00}{#1}}
\newcommand{\SpecialStringTok}[1]{\textcolor[rgb]{0.31,0.60,0.02}{#1}}
\newcommand{\StringTok}[1]{\textcolor[rgb]{0.31,0.60,0.02}{#1}}
\newcommand{\VariableTok}[1]{\textcolor[rgb]{0.00,0.00,0.00}{#1}}
\newcommand{\VerbatimStringTok}[1]{\textcolor[rgb]{0.31,0.60,0.02}{#1}}
\newcommand{\WarningTok}[1]{\textcolor[rgb]{0.56,0.35,0.01}{\textbf{\textit{#1}}}}
\usepackage{graphicx}
\makeatletter
\def\maxwidth{\ifdim\Gin@nat@width>\linewidth\linewidth\else\Gin@nat@width\fi}
\def\maxheight{\ifdim\Gin@nat@height>\textheight\textheight\else\Gin@nat@height\fi}
\makeatother
% Scale images if necessary, so that they will not overflow the page
% margins by default, and it is still possible to overwrite the defaults
% using explicit options in \includegraphics[width, height, ...]{}
\setkeys{Gin}{width=\maxwidth,height=\maxheight,keepaspectratio}
% Set default figure placement to htbp
\makeatletter
\def\fps@figure{htbp}
\makeatother
\setlength{\emergencystretch}{3em} % prevent overfull lines
\providecommand{\tightlist}{%
  \setlength{\itemsep}{0pt}\setlength{\parskip}{0pt}}
\setcounter{secnumdepth}{-\maxdimen} % remove section numbering
\ifLuaTeX
  \usepackage{selnolig}  % disable illegal ligatures
\fi
\IfFileExists{bookmark.sty}{\usepackage{bookmark}}{\usepackage{hyperref}}
\IfFileExists{xurl.sty}{\usepackage{xurl}}{} % add URL line breaks if available
\urlstyle{same} % disable monospaced font for URLs
\hypersetup{
  pdftitle={TP Statistiques},
  pdfauthor={Gabriel CANAPLE, BOYER THOMAS, LECLUSE Martin, FAYALA Mohamed},
  hidelinks,
  pdfcreator={LaTeX via pandoc}}

\title{TP Statistiques}
\author{Gabriel CANAPLE, BOYER THOMAS, LECLUSE Martin, FAYALA Mohamed}
\date{19 Janvier 2024}

\begin{document}
\maketitle

(décrire la régression, l'état des résidus, le test de normalité,
est-elle centrée, variance est constante ? indépendance ?) test de
shapiro, student, chi2 régression linéaire = chap 6 du cours y = ax + b
+ epsilon y = variable expliquée a, b = inconnues x = variable
explicative epsilon = résidus test de pertinence : a=0? test de biais :
b=0?

on peut aussi avoir plusieurs variables explicatives : y = a1\emph{x1 +
a2}x2 + \ldots{} + an*xn + epsilon

PENSER A définir le risque alpha pour chacun des tests (si on sait pas
quoi prendre prendre 5\%)

\hypertarget{introduction}{%
\section{Introduction}\label{introduction}}

L'objectif du TP est de comparer les performances d'algorithmes de
calcul de circuit hamiltonien dans un graphe, selon les critères de
longueur du chemin, et de temps d'exécution.

Nos conclusions se basent tout au long du TP sur les méthodes
statistiques aabordées en cours : test d'hypothèse, tests multiples,
régression linéaire.

Dans une première partie, nous montrerons une visualisation du problème
résolu par les algorithmes, puis nous comparerons ces derniers sur le
critère de la longueur du chemin hamiltonien trouvé, et ensuite sur le
temps d'exécution. Enfin, nous essayerons de proposer une régression
linéaire de la complexité de l'algorithme Branch \& Bound.

\hypertarget{visualisation-de-chemins}{%
\section{1. Visualisation de chemins}\label{visualisation-de-chemins}}

Voici les données présentes dans le fichier ``DonneesGPSvilles.csv'' :

\begin{Shaded}
\begin{Highlighting}[]
\FunctionTok{set.seed}\NormalTok{(}\DecValTok{35}\NormalTok{)}
\NormalTok{villes }\OtherTok{\textless{}{-}} \FunctionTok{read.csv}\NormalTok{(}\StringTok{\textquotesingle{}DonneesGPSvilles.csv\textquotesingle{}}\NormalTok{,}\AttributeTok{header=}\ConstantTok{TRUE}\NormalTok{,}\AttributeTok{dec=}\StringTok{\textquotesingle{}.\textquotesingle{}}\NormalTok{,}\AttributeTok{sep=}\StringTok{\textquotesingle{};\textquotesingle{}}\NormalTok{,}\AttributeTok{quote=}\StringTok{"}\SpecialCharTok{\textbackslash{}"}\StringTok{"}\NormalTok{)}
\FunctionTok{str}\NormalTok{(villes)}
\end{Highlighting}
\end{Shaded}

\begin{verbatim}
## 'data.frame':    22 obs. of  5 variables:
##  $ EU_circo : chr  "Sud-Est" "Sud-Est" "Nord-Ouest" "Est" ...
##  $ region   : chr  "Rhône-Alpes" "Corse" "Picardie" "Franche-Comté" ...
##  $ ville    : chr  "Lyon" "Ajaccio" "Amiens" "Besançon" ...
##  $ latitude : num  45.7 41.9 49.9 47.2 44.8 ...
##  $ longitude: num  4.847 8.733 2.3 6.033 -0.567 ...
\end{verbatim}

Et voici comment nous visualisaons les résultats données par les
algorithmes (ici, avec l'exemple de l'algorithme ``nearest'', comaré au
chemin optimal fourni) :

\begin{Shaded}
\begin{Highlighting}[]
\FunctionTok{set.seed}\NormalTok{(}\DecValTok{35}\NormalTok{)}
\NormalTok{coord }\OtherTok{\textless{}{-}} \FunctionTok{cbind}\NormalTok{(villes}\SpecialCharTok{$}\NormalTok{longitude,villes}\SpecialCharTok{$}\NormalTok{latitude)}
\NormalTok{dist }\OtherTok{\textless{}{-}} \FunctionTok{distanceGPS}\NormalTok{(coord)}
\NormalTok{voisins }\OtherTok{\textless{}{-}} \FunctionTok{TSPnearest}\NormalTok{(dist)}

\NormalTok{pathOpt }\OtherTok{\textless{}{-}} \FunctionTok{c}\NormalTok{(}\DecValTok{1}\NormalTok{,}\DecValTok{8}\NormalTok{,}\DecValTok{9}\NormalTok{,}\DecValTok{4}\NormalTok{,}\DecValTok{21}\NormalTok{,}\DecValTok{13}\NormalTok{,}\DecValTok{7}\NormalTok{,}\DecValTok{10}\NormalTok{,}\DecValTok{3}\NormalTok{,}\DecValTok{17}\NormalTok{,}\DecValTok{16}\NormalTok{,}\DecValTok{20}\NormalTok{,}\DecValTok{6}\NormalTok{,}\DecValTok{19}\NormalTok{,}\DecValTok{15}\NormalTok{,}\DecValTok{18}\NormalTok{,}\DecValTok{11}\NormalTok{,}\DecValTok{5}\NormalTok{,}\DecValTok{22}\NormalTok{,}\DecValTok{14}\NormalTok{,}\DecValTok{12}\NormalTok{,}\DecValTok{2}\NormalTok{)}

\FunctionTok{par}\NormalTok{(}\AttributeTok{mfrow=}\FunctionTok{c}\NormalTok{(}\DecValTok{1}\NormalTok{,}\DecValTok{2}\NormalTok{),}\AttributeTok{mar=}\FunctionTok{c}\NormalTok{(}\DecValTok{1}\NormalTok{,}\DecValTok{1}\NormalTok{,}\DecValTok{2}\NormalTok{,}\DecValTok{1}\NormalTok{))}
\FunctionTok{plotTrace}\NormalTok{(coord[voisins}\SpecialCharTok{$}\NormalTok{chemin,], }\AttributeTok{title=}\StringTok{\textquotesingle{}Nearest\textquotesingle{}}\NormalTok{)}
\FunctionTok{plotTrace}\NormalTok{(coord[pathOpt,], }\AttributeTok{title=}\StringTok{\textquotesingle{}Chemin optimal\textquotesingle{}}\NormalTok{)}
\end{Highlighting}
\end{Shaded}

\includegraphics{TPstat_CR_files/figure-latex/unnamed-chunk-2-1.pdf}

Le total de la longueurs des trajets (à vol d'oiseau) valent
respectivement, pour la méthode des plus proches voisins :

\begin{verbatim}
## [1] 4303.568
\end{verbatim}

et pour la méthode optimale :

\begin{verbatim}
## [1] 3793.06
\end{verbatim}

On remarque que le résultat de l'algorithme est peu performant, avec de
incohérences que l'on peut facilement voir sur la carte, et qui sont
confirmées par le total de la longueur des chemins.

Cela illustre l'intérêt de notre démarche : il faut pouvoir objectifier
la performance des algorithmes pour les comparer entre eux, et
sélectionner le meilleur (selon certains critères).

C'est l'objet des parties suivantes.

\hypertarget{comparaison-dalgorithmes}{%
\section{2. Comparaison d'algorithmes}\label{comparaison-dalgorithmes}}

On fixe le nombre de somets du graphe à 10. Les coordonnées cartésiennes
sont des lois uniformes sur {[}0,1{]} :

\begin{Shaded}
\begin{Highlighting}[]
\FunctionTok{set.seed}\NormalTok{(}\DecValTok{35}\NormalTok{)}
\NormalTok{n }\OtherTok{\textless{}{-}} \DecValTok{10} \CommentTok{\#nombre de noeuds}

\CommentTok{\#exemple de lancement unitaire}
\NormalTok{sommets }\OtherTok{\textless{}{-}} \FunctionTok{data.frame}\NormalTok{(}\AttributeTok{x =} \FunctionTok{runif}\NormalTok{(n), }\AttributeTok{y =} \FunctionTok{runif}\NormalTok{(n))}
\NormalTok{couts }\OtherTok{\textless{}{-}} \FunctionTok{distance}\NormalTok{(sommets)}
\FunctionTok{TSPsolve}\NormalTok{(dist,}\StringTok{\textquotesingle{}nearest\textquotesingle{}}\NormalTok{) }\CommentTok{\#LE RESULTAT ICI ME PARAIT BIZARRE (4000 alors que ya que des coordonnées entre 0 et 1, on est surs qu\textquotesingle{}on l\textquotesingle{}appelle avec le bon jeu de données ?)}
\end{Highlighting}
\end{Shaded}

\begin{verbatim}
## [1] 4303.568
\end{verbatim}

On lance 50 simulations par algorithme, ce qui donne le résultat suivant
(premières lignes uniquement) :

\begin{verbatim}
## [1] 2.633229 2.825005 2.770953 2.190799 2.912380
\end{verbatim}

\hypertarget{longueur-des-chemins}{%
\subsubsection{2.1. Longueur des chemins}\label{longueur-des-chemins}}

On s'intéresse ici à la longueur des chemins retournée par chacun des
méthodes. Notre but est de détermnier quel(s) algorithme(s) est le plus
performant selon ce critère, c'est-à-dire renvoie une longueur la plus
faible possible. Pour répondre à cette question, on visualisera d'abord
les résultats, avant de mener des études plus approfondies, d'abord sur
les algorithmes ``nearest'' et ``branch \& bound'', puis entre tous les
algortihmes 2 à 2.

\hypertarget{observation-des-ruxe9sultats}{%
\subsubsection{Observation des
résultats}\label{observation-des-ruxe9sultats}}

Voici d'abord une visualisation des résultats sous forme de boxplot :

\begin{Shaded}
\begin{Highlighting}[]
\FunctionTok{set.seed}\NormalTok{(}\DecValTok{35}\NormalTok{)}
\NormalTok{res2 }\OtherTok{\textless{}{-}} \FunctionTok{as.vector}\NormalTok{(res)}
\NormalTok{meth\_names }\OtherTok{\textless{}{-}} \FunctionTok{c}\NormalTok{(}\StringTok{\textquotesingle{}insertion\textquotesingle{}}\NormalTok{,}\StringTok{\textquotesingle{}repetitive\_nn\textquotesingle{}}\NormalTok{,}\StringTok{\textquotesingle{}two\_opt\textquotesingle{}}\NormalTok{,}\StringTok{\textquotesingle{}nearest\textquotesingle{}}\NormalTok{,}\StringTok{\textquotesingle{}branch\textquotesingle{}}\NormalTok{)}
\NormalTok{methods2 }\OtherTok{\textless{}{-}} \FunctionTok{rep}\NormalTok{(meth\_names,}\AttributeTok{each=}\NormalTok{nsimu) }

\FunctionTok{boxplot}\NormalTok{(res,}
        \AttributeTok{main=}\StringTok{"Résultats des algorithmes de calcul de circuit hamiltonien"}\NormalTok{,}
        \AttributeTok{xlab=}\StringTok{"Algorithmes"}\NormalTok{,}
        \AttributeTok{ylab=}\StringTok{"Longueur des chemins"}\NormalTok{)}
\end{Highlighting}
\end{Shaded}

\includegraphics{TPstat_CR_files/figure-latex/unnamed-chunk-7-1.pdf}

On observe des différences entre les résultats des algorithmes.
Celles-ci sont a priori assez failbes, puisque les espaces
interquartiles se chevauchent tous (cela est sûrement du aux petits
écrats entre les coordonnées). La moyenne de la longuer des chemins est
comprise entre 2.8 et 3.0 pour tous. Il semble tout de même que
``insertion'', ``repet\_nn'' et ``branch'' aient des résultats très
similaires, alors que ``two\_opt'' et ``nearest'' se démarquent du
reste, avec des résultats un peu plus élevés. Au sujet de la variance,
on observe que ``repet\_nn'' semble avoir une valeur faible, avec des
résultats des 1er et 9e déciles plus proche de la moyenne que pour les
autres algorithmes. ``nearest'', ``insertion,''two\_opt'' au contraire
paraissent avoir une variance plus élevée. Concernant ``branch'', la
dispersion des valeurs semblent surtout être vraie pour lse valeurs
faibles : le 9e décile est assez peu supérieur à la moyenne, alors que
le 1er décile y est très inférieur.

\hypertarget{test-de-normalituxe9}{%
\subsubsection{Test de normalité}\label{test-de-normalituxe9}}

Afin de vérifier si l'on peut faire des tests de Student avec les
résultats obtenus, nous vérifions qu'il satisfont tous l'hypothèse de
normalité avec un test de Shapiro :

\begin{Shaded}
\begin{Highlighting}[]
\FunctionTok{shapiro.test}\NormalTok{(res[,}\DecValTok{1}\NormalTok{])}
\end{Highlighting}
\end{Shaded}

\begin{verbatim}
## 
##  Shapiro-Wilk normality test
## 
## data:  res[, 1]
## W = 0.97675, p-value = 0.4246
\end{verbatim}

\begin{Shaded}
\begin{Highlighting}[]
\FunctionTok{shapiro.test}\NormalTok{(res[,}\DecValTok{2}\NormalTok{])}
\end{Highlighting}
\end{Shaded}

\begin{verbatim}
## 
##  Shapiro-Wilk normality test
## 
## data:  res[, 2]
## W = 0.97601, p-value = 0.3983
\end{verbatim}

\begin{Shaded}
\begin{Highlighting}[]
\FunctionTok{shapiro.test}\NormalTok{(res[,}\DecValTok{3}\NormalTok{])}
\end{Highlighting}
\end{Shaded}

\begin{verbatim}
## 
##  Shapiro-Wilk normality test
## 
## data:  res[, 3]
## W = 0.95152, p-value = 0.03939
\end{verbatim}

\begin{Shaded}
\begin{Highlighting}[]
\FunctionTok{shapiro.test}\NormalTok{(res[,}\DecValTok{4}\NormalTok{])}
\end{Highlighting}
\end{Shaded}

\begin{verbatim}
## 
##  Shapiro-Wilk normality test
## 
## data:  res[, 4]
## W = 0.96797, p-value = 0.1911
\end{verbatim}

\begin{Shaded}
\begin{Highlighting}[]
\FunctionTok{shapiro.test}\NormalTok{(res[,}\DecValTok{5}\NormalTok{])}
\end{Highlighting}
\end{Shaded}

\begin{verbatim}
## 
##  Shapiro-Wilk normality test
## 
## data:  res[, 5]
## W = 0.95604, p-value = 0.06067
\end{verbatim}

Avec un seuil de risque à 5\%, on ne rejette pas l'hypothèse de
normalité pour toutes les distributions. On peut donc faire des tests de
Student sur ces résultats.

\hypertarget{comparaison-de-nearest-et-de-branch-bound}{%
\subsubsection{Comparaison de ``nearest'' et de ``branch \&
bound''}\label{comparaison-de-nearest-et-de-branch-bound}}

Notre premier objectif est de comparer les performances de ``nearest''
et de ``branch \& bound''. En effet, ce sont les algorithmes qui ont a
priori le plus de chances d'être significativement différents (cf
boxplot).

On cherche à savoir si ``nearest'' est significativement moins
performant que ``branch \& bound'', donc on pose les hypothèse suivantes
\(H(0) : m_n - m_bb <= 0\) et \(H(1) : m_n-m_bb > 0\). De cette manière,
on contrôle le risque de conclure que ``nearest'' est meilleur que
``branch \& bound'' alors que ce n'est pas le cas. A noter que, ici,
l'algorithme le plus performant trouve une longueur plus faible.

Pour ce test, on prend un seuil de risque de 5\%.

Précisons également que la soustraction \(m_n _ m_bb\) est faite car
nous avons ici des calculs faits sur le même échantillon, et qu'elle
nous peremt de limiter les incertitudes : en effet, on construit une
troisième gaussienne (la différence des deux premières), que l'on
compare à une valeur constante, plutôt que de comparer deux gaussiennes
entre elles.

\begin{Shaded}
\begin{Highlighting}[]
\NormalTok{nearest\_branch }\OtherTok{\textless{}{-}}\NormalTok{ res[,}\DecValTok{4}\NormalTok{] }\SpecialCharTok{{-}}\NormalTok{ res[,}\DecValTok{5}\NormalTok{]}
\FunctionTok{t.test}\NormalTok{(res[,}\DecValTok{4}\NormalTok{], res[,}\DecValTok{5}\NormalTok{], }\AttributeTok{alternative =} \StringTok{"greater"}\NormalTok{)}
\end{Highlighting}
\end{Shaded}

\begin{verbatim}
## 
##  Welch Two Sample t-test
## 
## data:  res[, 4] and res[, 5]
## t = 4.0462, df = 97.621, p-value = 5.211e-05
## alternative hypothesis: true difference in means is greater than 0
## 95 percent confidence interval:
##  0.1814195       Inf
## sample estimates:
## mean of x mean of y 
##  3.119036  2.811328
\end{verbatim}

Avec une p-valeur de 0.00005\%, on conclut qu'on rejette l'hypothèse
nulle, et que \(m_n\) est significativement moins performant que
\(m_bb\). De plus, la très faible p-valeur nous donne une grande
confiance dans ce résultat.

\hypertarget{tests-deux-uxe0-deux}{%
\subsubsection{Tests deux à deux}\label{tests-deux-uxe0-deux}}

L'objectif de cette analyse est de déterminer si l'on peut regrouper les
différents algorithmes par performance sur le critère de longueur des
chemins trouvés. Cela permettra de présenter les résultats sous une
forme plus intelligible, et éventuellement de faire plus tard des choix
entre les algorithmes pus facilement (en prenant d'autres critères comme
le temps d'exécution par exemple).

\begin{Shaded}
\begin{Highlighting}[]
\NormalTok{result }\OtherTok{\textless{}{-}} \FunctionTok{pairwise.t.test}\NormalTok{(res2,methods2, }\AttributeTok{p.adjust.method =} \StringTok{"bonferroni"}\NormalTok{)}
\NormalTok{result}
\end{Highlighting}
\end{Shaded}

\begin{verbatim}
## 
##  Pairwise comparisons using t tests with pooled SD 
## 
## data:  res2 and methods2 
## 
##               branch  insertion nearest repetitive_nn
## insertion     1.00000 -         -       -            
## nearest       0.00052 0.03281   -       -            
## repetitive_nn 1.00000 1.00000   0.07124 -            
## two_opt       0.00441 0.16646   1.00000 0.32098      
## 
## P value adjustment method: bonferroni
\end{verbatim}

Nous prenons un risque \(\alpha\) égal à 5\%. La seule différence
notable est entre nearest et branch, avec une p-valeur inférieure à 5\%.
Nous avons choisi de répartir les algorithmes entre 4 groupes : - Le
premier groupe est composé d'insertion, et de repetitive-nn. - Le
deuxième groupe se constitue de nearest. - Le troisième groupe contient
two-opt. - Le quatrième groupe contient branch and bound La philosophie
de cette répartition est de rassembler les algorithmes similaires entre
eux, et ayant des différences avec les même algorithmes. A compléter.

\hypertarget{temps-de-calcul}{%
\subsection{2.2. Temps de calcul}\label{temps-de-calcul}}

Après avoir étudié les performances des algorithmes selon la longueur
des chemins trouvés, nous étudions les performances en termes de temps
de calcul.

On exécute les algorithmes 20 fois, les coordonnées des points sont
comme précédemment générés par une lois uniforme sur {[}0,1{]}.

On utilise pour cela le package ``microbenchmark'' :

\begin{Shaded}
\begin{Highlighting}[]
\FunctionTok{set.seed}\NormalTok{(}\DecValTok{35}\NormalTok{)}

\NormalTok{microbenchmark}\SpecialCharTok{::}\FunctionTok{microbenchmark}\NormalTok{(}\FunctionTok{TSPsolve}\NormalTok{(jeuDeDonnees, }\AttributeTok{method=}\NormalTok{methods[}\DecValTok{1}\NormalTok{]),}\FunctionTok{TSPsolve}\NormalTok{(jeuDeDonnees, }\AttributeTok{method=}\NormalTok{methods[}\DecValTok{2}\NormalTok{]), }\FunctionTok{TSPsolve}\NormalTok{(jeuDeDonnees, }\AttributeTok{method=}\NormalTok{methods[}\DecValTok{3}\NormalTok{]), }\FunctionTok{TSPsolve}\NormalTok{(jeuDeDonnees, }\AttributeTok{method=}\NormalTok{methods[}\DecValTok{4}\NormalTok{]), }\FunctionTok{TSPsolve}\NormalTok{(jeuDeDonnees, }\AttributeTok{method=}\NormalTok{methods[}\DecValTok{5}\NormalTok{]), }\AttributeTok{times=}\DecValTok{20}\NormalTok{, }\AttributeTok{setup=}\NormalTok{\{jeuDeDonnees }\OtherTok{\textless{}{-}} \FunctionTok{distance}\NormalTok{(}\FunctionTok{data.frame}\NormalTok{(}\AttributeTok{x =} \FunctionTok{runif}\NormalTok{(n), }\AttributeTok{y =} \FunctionTok{runif}\NormalTok{(n)))\})}
\end{Highlighting}
\end{Shaded}

\begin{verbatim}
## Unit: microseconds
##                                         expr    min      lq      mean  median
##  TSPsolve(jeuDeDonnees, method = methods[1])  932.7 1030.25  1179.435 1088.15
##  TSPsolve(jeuDeDonnees, method = methods[2]) 9348.4 9413.10 10216.235 9685.35
##  TSPsolve(jeuDeDonnees, method = methods[3])  887.6  907.60  1020.095  987.80
##  TSPsolve(jeuDeDonnees, method = methods[4])   19.2   28.70    32.860   31.00
##  TSPsolve(jeuDeDonnees, method = methods[5]) 2902.0 6078.40 10289.680 9841.70
##        uq     max neval cld
##   1232.55  1864.8    20  a 
##  11060.50 13250.0    20   b
##   1015.30  1888.3    20  a 
##     37.35    53.6    20  a 
##  13133.75 23669.6    20   b
\end{verbatim}

Les lignes correspondent dans l'ordre à : ``insertion'', ``repet\_nn'',
``two\_opt'', ``nearest'' et ``branch''.

Les résultat permettent de séparer les algorithmes en 2 classe : -
``insertion'', ``two\_opt'' et ``nearest'' : exécution plus rapide, avec
une moyenne comprise entre 7 et 260 microsecondes - ``repet\_nn'' et
``branch \& bound'' : exécution plus lente, avec une moyenne supérieur à
2200 microsecondes

Si l'on compare avec les résultats de la partie précédente, et la
longuer des chemins, on

\hypertarget{etude-de-la-complexituxe9-de-lalgorithme-branch-and-bound}{%
\section{2. Etude de la complexité de l'algorithme Branch and
Bound}\label{etude-de-la-complexituxe9-de-lalgorithme-branch-and-bound}}

Dans cette partie, on étudie la complexité de Branch \& Bound, en
étudiant le temps d'exécution en fonction de la taille du graphe.

\hypertarget{comportement-par-rapport-au-nombre-de-sommets-premier-moduxe8le}{%
\subsection{2.1. Comportement par rapport au nombre de sommets : premier
modèle}\label{comportement-par-rapport-au-nombre-de-sommets-premier-moduxe8le}}

On crée à chaque fois 10 graphes, pour des valeurs de \(n\) entre 4 et
20 inclus. Comme précédemment, on généère les coordonnées des points
avec une loi uniforme sur {[}0,1{]} :

\begin{Shaded}
\begin{Highlighting}[]
\FunctionTok{set.seed}\NormalTok{(}\DecValTok{35}\NormalTok{)}
\NormalTok{sommets }\OtherTok{\textless{}{-}} \FunctionTok{data.frame}\NormalTok{(}\AttributeTok{x =} \FunctionTok{runif}\NormalTok{(n), }\AttributeTok{y =} \FunctionTok{runif}\NormalTok{(n))}
\NormalTok{couts }\OtherTok{\textless{}{-}} \FunctionTok{distance}\NormalTok{(sommets)}
\end{Highlighting}
\end{Shaded}

Nous construisons un modèle de régression linéaire simple du temps
d'exécution de Branch\&Bound en fonction du nombre de sommets n.
Introduisons

\begin{Shaded}
\begin{Highlighting}[]
\FunctionTok{set.seed}\NormalTok{(}\DecValTok{35}\NormalTok{)}
\NormalTok{seqn }\OtherTok{\textless{}{-}} \FunctionTok{seq}\NormalTok{(}\DecValTok{4}\NormalTok{,}\DecValTok{20}\NormalTok{,}\DecValTok{1}\NormalTok{)}
\end{Highlighting}
\end{Shaded}

On construit la matrice des résultats :

\begin{Shaded}
\begin{Highlighting}[]
\FunctionTok{set.seed}\NormalTok{(}\DecValTok{35}\NormalTok{)}
\NormalTok{temps  }\OtherTok{\textless{}{-}} \FunctionTok{matrix}\NormalTok{(}\AttributeTok{nrow =} \FunctionTok{length}\NormalTok{(seqn), }\AttributeTok{ncol=}\DecValTok{10}\NormalTok{)}
\ControlFlowTok{for}\NormalTok{ (i }\ControlFlowTok{in} \DecValTok{1}\SpecialCharTok{:}\DecValTok{17}\NormalTok{) \{}
\NormalTok{  temps[i,] }\OtherTok{=} 
  \FunctionTok{microbenchmark}\NormalTok{(}\FunctionTok{TSPsolve}\NormalTok{(couts, }\AttributeTok{method =} \StringTok{\textquotesingle{}branch\textquotesingle{}}\NormalTok{),}
  \AttributeTok{times =} \DecValTok{10}\NormalTok{,}
  \AttributeTok{setup =}\NormalTok{ \{ n }\OtherTok{\textless{}{-}}\NormalTok{ seqn[i]}
\NormalTok{  couts }\OtherTok{\textless{}{-}} \FunctionTok{distance}\NormalTok{(}\FunctionTok{cbind}\NormalTok{(}\AttributeTok{x =} \FunctionTok{runif}\NormalTok{(n), }\AttributeTok{y =} \FunctionTok{runif}\NormalTok{(n))) \}}
\NormalTok{  )}\SpecialCharTok{$}\NormalTok{time}
\NormalTok{\}}
\end{Highlighting}
\end{Shaded}

Et on affiche les résultats sur un graphe. Les premiers résultats nous
laissant penser à une relations de forme \(exp(n/2)\), on affiche aussi
\(\log(temps)^2\) :

\begin{Shaded}
\begin{Highlighting}[]
\FunctionTok{set.seed}\NormalTok{(}\DecValTok{35}\NormalTok{)}
\FunctionTok{par}\NormalTok{(}\AttributeTok{mfrow=}\FunctionTok{c}\NormalTok{(}\DecValTok{1}\NormalTok{,}\DecValTok{2}\NormalTok{)) }\CommentTok{\# 2 graphiques sur 1 ligne}
\FunctionTok{matplot}\NormalTok{(seqn, temps, }\AttributeTok{xlab=}\StringTok{\textquotesingle{}n\textquotesingle{}}\NormalTok{, }\AttributeTok{ylab=}\StringTok{\textquotesingle{}temps\textquotesingle{}}\NormalTok{)}
\FunctionTok{matplot}\NormalTok{(seqn, }\FunctionTok{log}\NormalTok{(temps)}\SpecialCharTok{\^{}}\DecValTok{2}\NormalTok{, }\AttributeTok{xlab=}\StringTok{\textquotesingle{}n\textquotesingle{}}\NormalTok{, }\AttributeTok{ylab=}\FunctionTok{expression}\NormalTok{(}\FunctionTok{log}\NormalTok{(temps)}\SpecialCharTok{\^{}}\DecValTok{2}\NormalTok{))}
\end{Highlighting}
\end{Shaded}

\includegraphics{TPstat_CR_files/figure-latex/unnamed-chunk-15-1.pdf}

L'observation des résultats nous permet de suspecter une relation
linéaire, même si le deuxième graphe montre une relation qui semble
légèrement logarithmique.

On crée le modèle de régression linéaire de \(\log(temps)^2\) en
fonction de \(n\) :

\begin{Shaded}
\begin{Highlighting}[]
\FunctionTok{set.seed}\NormalTok{(}\DecValTok{35}\NormalTok{)}
\NormalTok{vect\_temps }\OtherTok{\textless{}{-}} \FunctionTok{log}\NormalTok{(}\FunctionTok{as.vector}\NormalTok{(temps))}\SpecialCharTok{\^{}}\DecValTok{2}
\NormalTok{vect\_dim }\OtherTok{\textless{}{-}} \FunctionTok{rep}\NormalTok{(seqn, }\AttributeTok{times=}\DecValTok{10}\NormalTok{)}
\NormalTok{temps.lm }\OtherTok{\textless{}{-}} \FunctionTok{lm}\NormalTok{(vect\_temps }\SpecialCharTok{\textasciitilde{}}\NormalTok{ vect\_dim)}
\FunctionTok{summary}\NormalTok{(temps.lm)}
\end{Highlighting}
\end{Shaded}

\begin{verbatim}
## 
## Call:
## lm(formula = vect_temps ~ vect_dim)
## 
## Residuals:
##     Min      1Q  Median      3Q     Max 
## -69.160 -22.908   1.237  21.986  67.103 
## 
## Coefficients:
##             Estimate Std. Error t value Pr(>|t|)    
## (Intercept)  84.0325     5.9954   14.02   <2e-16 ***
## vect_dim     15.2465     0.4626   32.96   <2e-16 ***
## ---
## Signif. codes:  0 '***' 0.001 '**' 0.01 '*' 0.05 '.' 0.1 ' ' 1
## 
## Residual standard error: 29.55 on 168 degrees of freedom
## Multiple R-squared:  0.8661, Adjusted R-squared:  0.8653 
## F-statistic:  1086 on 1 and 168 DF,  p-value: < 2.2e-16
\end{verbatim}

\hypertarget{test-de-pertinence}{%
\subsubsection{Test de pertinence}\label{test-de-pertinence}}

Ici, la p-valeur de l'hypothèse \(a=0\) (ligne vect\_dim) est
extrêmement inférieure à 5\%, ainsi, l'hypothèse \(a=0\) est rejetée
avec une grande confiance. Le modèle linéaire est pertinent.

\hypertarget{etude-du-biais}{%
\subsubsection{Etude du biais}\label{etude-du-biais}}

Ici, la p-valeur de l'hypothèse \(b=0\) (ligne vect\_dim) est
extrêmement inférieure à 5\%, ainsi, l'hypothèse \(b=0\) est rejetée
avec une grande confiance : il y a un biais dans la relation.

\hypertarget{etude-des-ruxe9sidus}{%
\subsubsection{Etude des résidus}\label{etude-des-ruxe9sidus}}

\hypertarget{test-de-normalituxe9-1}{%
\paragraph{Test de normalité}\label{test-de-normalituxe9-1}}

Comme précédemment, le test de normalité est per le test de Shapiro,
avec un seuil de risque de 5\% :

\begin{Shaded}
\begin{Highlighting}[]
\FunctionTok{shapiro.test}\NormalTok{(}\FunctionTok{residuals}\NormalTok{(temps.lm))}
\end{Highlighting}
\end{Shaded}

\begin{verbatim}
## 
##  Shapiro-Wilk normality test
## 
## data:  residuals(temps.lm)
## W = 0.98752, p-value = 0.1362
\end{verbatim}

On obtient une p-valeur = 0.37 \textgreater{} 5\%. On ne rejette pas
l'hypothèse de normalité : les résidus suivent une loi normale.

\hypertarget{etude-graphique}{%
\subparagraph{Etude graphique}\label{etude-graphique}}

\begin{Shaded}
\begin{Highlighting}[]
\FunctionTok{par}\NormalTok{(}\AttributeTok{mfrow=}\FunctionTok{c}\NormalTok{(}\DecValTok{1}\NormalTok{,}\DecValTok{2}\NormalTok{)) }\CommentTok{\# 4 graphiques sur 2 lignes et 2 colonnes}
\FunctionTok{plot}\NormalTok{(temps.lm)}
\end{Highlighting}
\end{Shaded}

\includegraphics{TPstat_CR_files/figure-latex/unnamed-chunk-18-1.pdf}
\includegraphics{TPstat_CR_files/figure-latex/unnamed-chunk-18-2.pdf}
Résiduals vs Fitted : la courbe n'est ni horizontale ni homogène. Normal
Q-Q : L'ensemble des points sont sur la diagonale avec quelques
exceptions (points 135 et 46). On peut en désuire que les résidus
suivent une loi normale\\
Scale location : courbe moins concave que Résiduals vs Fitted mais
toujour pas horizontale Residuals vs Leverage : les points sont éloignés
de la distance de Cook = 1

\hypertarget{tests}{%
\subparagraph{Tests}\label{tests}}

\hypertarget{test-despuxe9rance}{%
\paragraph{Test d'espérance}\label{test-despuxe9rance}}

\hypertarget{etude-graphique-1}{%
\subparagraph{Etude graphique}\label{etude-graphique-1}}

\hypertarget{tests-1}{%
\subparagraph{Tests}\label{tests-1}}

\hypertarget{test-de-variance}{%
\paragraph{Test de variance}\label{test-de-variance}}

\hypertarget{etude-graphique-2}{%
\subparagraph{Etude graphique}\label{etude-graphique-2}}

\hypertarget{tests-2}{%
\subparagraph{Tests}\label{tests-2}}

\hypertarget{test-dinduxe9pendance}{%
\paragraph{Test d'indépendance}\label{test-dinduxe9pendance}}

\hypertarget{etude-graphique-3}{%
\subparagraph{Etude graphique}\label{etude-graphique-3}}

\hypertarget{comportement-par-rapport-au-nombre-de-sommets-uxe9tude-du-comportement-moyen}{%
\subsection{2.2. Comportement par rapport au nombre de sommets : étude
du comportement
moyen}\label{comportement-par-rapport-au-nombre-de-sommets-uxe9tude-du-comportement-moyen}}

Récupération du temps moyen.

Ajustement du modèle linéaire de \(\log(temps.moy)^2\) en fonction de
\(n\).

\begin{Shaded}
\begin{Highlighting}[]
\NormalTok{temps.moy }\OtherTok{\textless{}{-}} \FunctionTok{rowMeans}\NormalTok{(temps)}
\FunctionTok{matplot}\NormalTok{(seqn, }\FunctionTok{log}\NormalTok{(temps.moy)}\SpecialCharTok{\^{}}\DecValTok{2}\NormalTok{, }\AttributeTok{xlab=}\StringTok{\textquotesingle{}n\textquotesingle{}}\NormalTok{, }\AttributeTok{ylab=}\FunctionTok{expression}\NormalTok{(}\FunctionTok{log}\NormalTok{(temps.moy)}\SpecialCharTok{\^{}}\DecValTok{2}\NormalTok{))}
\end{Highlighting}
\end{Shaded}

\includegraphics{TPstat_CR_files/figure-latex/unnamed-chunk-19-1.pdf}

Analyse de la validité du modèle :

\begin{itemize}
\tightlist
\item
  pertinence des coefficients et du modèle,
\end{itemize}

\begin{Shaded}
\begin{Highlighting}[]
\NormalTok{vect\_moy }\OtherTok{\textless{}{-}} \FunctionTok{log}\NormalTok{(}\FunctionTok{as.vector}\NormalTok{(temps.moy))}\SpecialCharTok{\^{}}\DecValTok{2}
\NormalTok{vect\_dim\_moy }\OtherTok{\textless{}{-}} \FunctionTok{rep}\NormalTok{(seqn)}
\NormalTok{temps.moy.lm }\OtherTok{\textless{}{-}} \FunctionTok{lm}\NormalTok{(vect\_dim\_moy }\SpecialCharTok{\textasciitilde{}}\NormalTok{ vect\_moy)}
\FunctionTok{summary}\NormalTok{(temps.moy.lm)}
\end{Highlighting}
\end{Shaded}

\begin{verbatim}
## 
## Call:
## lm(formula = vect_dim_moy ~ vect_moy)
## 
## Residuals:
##     Min      1Q  Median      3Q     Max 
## -2.4506 -1.0172 -0.2980  0.9404  3.0612 
## 
## Coefficients:
##              Estimate Std. Error t value Pr(>|t|)    
## (Intercept) -3.981057   1.327564  -2.999    0.009 ** 
## vect_moy     0.058762   0.004685  12.544 2.35e-09 ***
## ---
## Signif. codes:  0 '***' 0.001 '**' 0.01 '*' 0.05 '.' 0.1 ' ' 1
## 
## Residual standard error: 1.539 on 15 degrees of freedom
## Multiple R-squared:  0.913,  Adjusted R-squared:  0.9072 
## F-statistic: 157.3 on 1 and 15 DF,  p-value: 2.354e-09
\end{verbatim}

\begin{itemize}
\tightlist
\item
  étude des hypothèses sur les résidus.
\end{itemize}

\begin{Shaded}
\begin{Highlighting}[]
\FunctionTok{par}\NormalTok{(}\AttributeTok{mfrow=}\FunctionTok{c}\NormalTok{(}\DecValTok{1}\NormalTok{,}\DecValTok{2}\NormalTok{)) }\CommentTok{\# 4 graphiques sur 2 lignes et 2 colonnes}
\FunctionTok{plot}\NormalTok{(temps.moy.lm)}
\end{Highlighting}
\end{Shaded}

\includegraphics{TPstat_CR_files/figure-latex/unnamed-chunk-21-1.pdf}
\includegraphics{TPstat_CR_files/figure-latex/unnamed-chunk-21-2.pdf}

\begin{Shaded}
\begin{Highlighting}[]
\FunctionTok{shapiro.test}\NormalTok{(}\FunctionTok{residuals}\NormalTok{(temps.moy.lm))}
\end{Highlighting}
\end{Shaded}

\begin{verbatim}
## 
##  Shapiro-Wilk normality test
## 
## data:  residuals(temps.moy.lm)
## W = 0.96387, p-value = 0.7049
\end{verbatim}

\hypertarget{comportement-par-rapport-uxe0-la-structure-du-graphe}{%
\subsection{2.3. Comportement par rapport à la structure du
graphe}\label{comportement-par-rapport-uxe0-la-structure-du-graphe}}

Lecture du fichier `DonneesTSP.csv'.

\begin{Shaded}
\begin{Highlighting}[]
\NormalTok{data.graph }\OtherTok{\textless{}{-}} \FunctionTok{read.csv}\NormalTok{(}\AttributeTok{file=}\StringTok{\textquotesingle{}DonneesTSP.csv\textquotesingle{}}\NormalTok{,}\AttributeTok{header=}\ConstantTok{TRUE}\NormalTok{)}
\FunctionTok{str}\NormalTok{(data.graph)}
\end{Highlighting}
\end{Shaded}

\begin{verbatim}
## 'data.frame':    70 obs. of  8 variables:
##  $ tps      : num  53692 144081 997803 2553322 6333009 ...
##  $ dim      : int  4 6 8 10 12 14 16 18 20 4 ...
##  $ mean.long: num  0.391 0.442 0.334 0.276 0.254 ...
##  $ mean.dist: num  0.665 0.592 0.537 0.506 0.502 ...
##  $ sd.dist  : num  0.276 0.259 0.246 0.238 0.227 ...
##  $ mean.deg : num  3 5 7 9 11 13 15 17 19 3 ...
##  $ sd.deg   : num  0 0 0 0 0 0 0 0 0 0 ...
##  $ diameter : num  1 1 1 1 1 1 1 1 1 1 ...
\end{verbatim}

Ajustement du modèle linéaire de \(\log(temps.moy)^2\) en fonction de
toutes les variables présentes. Modèle sans constante.

\begin{Shaded}
\begin{Highlighting}[]
\NormalTok{data.graph}\SpecialCharTok{$}\NormalTok{log.tps }\OtherTok{\textless{}{-}} \FunctionTok{log}\NormalTok{(data.graph}\SpecialCharTok{$}\NormalTok{tps)}\CommentTok{\#log(donnees$tps)\^{}2}
\NormalTok{data.graph}\SpecialCharTok{$}\NormalTok{sqrt.dim }\OtherTok{\textless{}{-}} \FunctionTok{sqrt}\NormalTok{(data.graph}\SpecialCharTok{$}\NormalTok{dim)}
\NormalTok{data.graph}\SpecialCharTok{$}\NormalTok{tps }\OtherTok{\textless{}{-}} \FunctionTok{c}\NormalTok{() }\CommentTok{\#on retire les variables tps et dim devenues inutiles}
\NormalTok{data.graph}\SpecialCharTok{$}\NormalTok{dim }\OtherTok{\textless{}{-}} \FunctionTok{c}\NormalTok{()}
\FunctionTok{str}\NormalTok{(data.graph)}
\end{Highlighting}
\end{Shaded}

\begin{verbatim}
## 'data.frame':    70 obs. of  8 variables:
##  $ mean.long: num  0.391 0.442 0.334 0.276 0.254 ...
##  $ mean.dist: num  0.665 0.592 0.537 0.506 0.502 ...
##  $ sd.dist  : num  0.276 0.259 0.246 0.238 0.227 ...
##  $ mean.deg : num  3 5 7 9 11 13 15 17 19 3 ...
##  $ sd.deg   : num  0 0 0 0 0 0 0 0 0 0 ...
##  $ diameter : num  1 1 1 1 1 1 1 1 1 1 ...
##  $ log.tps  : num  10.9 11.9 13.8 14.8 15.7 ...
##  $ sqrt.dim : num  2 2.45 2.83 3.16 3.46 ...
\end{verbatim}

\begin{Shaded}
\begin{Highlighting}[]
\NormalTok{modele.complet }\OtherTok{=} \FunctionTok{lm}\NormalTok{(data.graph}\SpecialCharTok{$}\NormalTok{log.tps}\SpecialCharTok{\textasciitilde{}}\NormalTok{., }\AttributeTok{data =}\NormalTok{ data.graph)}
\end{Highlighting}
\end{Shaded}

Mise en \oe uvre d'une sélection de variables pour ne garder que les
variables pertinentes.

Analyse de la validité du modèle :

\begin{Shaded}
\begin{Highlighting}[]
\FunctionTok{step}\NormalTok{(modele.complet)}
\end{Highlighting}
\end{Shaded}

\begin{verbatim}
## Start:  AIC=-165.23
## data.graph$log.tps ~ mean.long + mean.dist + sd.dist + mean.deg + 
##     sd.deg + diameter + sqrt.dim
## 
##             Df Sum of Sq     RSS      AIC
## - diameter   1    0.0145  5.2711 -167.038
## <none>                    5.2566 -165.230
## - sd.deg     1    0.2182  5.4748 -164.384
## - mean.dist  1    0.3014  5.5581 -163.327
## - mean.deg   1    0.8757  6.1324 -156.444
## - mean.long  1    3.6951  8.9517 -129.965
## - sd.dist    1    4.4335  9.6902 -124.417
## - sqrt.dim   1   17.3311 22.5877  -65.176
## 
## Step:  AIC=-167.04
## data.graph$log.tps ~ mean.long + mean.dist + sd.dist + mean.deg + 
##     sd.deg + sqrt.dim
## 
##             Df Sum of Sq     RSS      AIC
## <none>                    5.2711 -167.038
## - sd.deg     1    0.2065  5.4776 -166.349
## - mean.dist  1    0.6554  5.9265 -160.835
## - mean.deg   1    0.9820  6.2531 -157.080
## - mean.long  1    3.8220  9.0931 -130.869
## - sd.dist    1    4.9133 10.1844 -122.935
## - sqrt.dim   1   18.7788 24.0499  -62.785
\end{verbatim}

\begin{verbatim}
## 
## Call:
## lm(formula = data.graph$log.tps ~ mean.long + mean.dist + sd.dist + 
##     mean.deg + sd.deg + sqrt.dim, data = data.graph)
## 
## Coefficients:
## (Intercept)    mean.long    mean.dist      sd.dist     mean.deg       sd.deg  
##    6.396008    -4.854857    -0.002284     0.004883    -0.140823     0.126916  
##    sqrt.dim  
##    3.444077
\end{verbatim}

\begin{itemize}
\item
  pertinence des coefficients et du modèle,
\item
  étude des hypothèses sur les résidus.
\end{itemize}

\begin{Shaded}
\begin{Highlighting}[]
\FunctionTok{par}\NormalTok{(}\AttributeTok{mfrow=}\FunctionTok{c}\NormalTok{(}\DecValTok{1}\NormalTok{,}\DecValTok{2}\NormalTok{)) }\CommentTok{\# 4 graphiques sur 2 lignes et 2 colonnes}
\FunctionTok{plot}\NormalTok{(modele.complet)}
\end{Highlighting}
\end{Shaded}

\includegraphics{TPstat_CR_files/figure-latex/unnamed-chunk-27-1.pdf}
\includegraphics{TPstat_CR_files/figure-latex/unnamed-chunk-27-2.pdf}

\begin{Shaded}
\begin{Highlighting}[]
\FunctionTok{shapiro.test}\NormalTok{(}\FunctionTok{residuals}\NormalTok{(modele.complet))}
\end{Highlighting}
\end{Shaded}

\begin{verbatim}
## 
##  Shapiro-Wilk normality test
## 
## data:  residuals(modele.complet)
## W = 0.98289, p-value = 0.455
\end{verbatim}

\end{document}
